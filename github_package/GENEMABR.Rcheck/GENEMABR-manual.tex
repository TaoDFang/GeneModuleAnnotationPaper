\nonstopmode{}
\documentclass[a4paper]{book}
\usepackage[times,inconsolata,hyper]{Rd}
\usepackage{makeidx}
\usepackage[utf8]{inputenc} % @SET ENCODING@
% \usepackage{graphicx} % @USE GRAPHICX@
\makeindex{}
\begin{document}
\chapter*{}
\begin{center}
{\textbf{\huge Package `GENEMABR'}}
\par\bigskip{\large \today}
\end{center}
\begin{description}
\raggedright{}
\inputencoding{utf8}
\item[Title]\AsIs{Gene module/list annotation}
\item[Version]\AsIs{0.99.0}
\item[Author]\AsIs{Tao Fang, Daniel Marbach, Zhang, Jitao David}
\item[Maintainer]\AsIs{Tao Fang }\email{talonvonfang@gmail.com}\AsIs{}
\item[Description]\AsIs{Gene-set module annotation or gene-set enrichment within a regression based framework.}
\item[Imports]\AsIs{glmnet, igraph, Matrix,}
\item[Depends]\AsIs{R (>= 3.6)}
\item[biocViews]\AsIs{GeneSetEnrichment,Regression,Pathways,GO, Reactome}
\item[License]\AsIs{Artistic-2.0}
\item[BugReports]\AsIs{}\url{https://github.com/TaoDFang/GENEMABR/issues}\AsIs{}
\item[LazyData]\AsIs{false}
\item[Encoding]\AsIs{UTF-8}
\item[VignetteBuilder]\AsIs{knitr}
\item[Suggests]\AsIs{BiocStyle, knitr, rmarkdown, testthat (>= 2.1.0)}
\item[RoxygenNote]\AsIs{6.1.1}
\item[NeedsCompilation]\AsIs{no}
\end{description}
\Rdcontents{\R{} topics documented:}
\inputencoding{utf8}
\HeaderA{find\_root\_ids}{find\_root\_ids}{find.Rul.root.Rul.ids}
%
\begin{Description}\relax
If you use the default pathway databases(GO Ontologyand REACTOME),this function allows you to extract GO sub-roots or REACTOME roots for certain pathways from GO or REACTOME
to help you better understanding thier the biological meanings
\end{Description}
%
\begin{Usage}
\begin{verbatim}
find_root_ids(selected_pathways)
\end{verbatim}
\end{Usage}
%
\begin{Arguments}
\begin{ldescription}
\item[\code{selected\_pathways}] A vecor of GO and/or REACTOME pathways IDs.
\end{ldescription}
\end{Arguments}
%
\begin{Value}
A list of GO sub-root or REACTOME root ids for provided pathways.
If a certain pathway has morn than one GO sub-roots or REACTOME roots, they will be seperated by "\#".
\end{Value}
%
\begin{Examples}
\begin{ExampleCode}
find_root_ids(selected_pathways=c("GO:0005834","R-HSA-111469"))
\end{ExampleCode}
\end{Examples}
\inputencoding{utf8}
\HeaderA{fisher\_exact\_test}{fisher\_exact\_test}{fisher.Rul.exact.Rul.test}
%
\begin{Description}\relax
This function allows you to compute two sided fish exact pvalue of gene list for selected  pathways
To know more about this method. I recommend you to read the paper (Enrichment or depletion of a GO category within a class of genes: which test?) for more details
\end{Description}
%
\begin{Usage}
\begin{verbatim}
fisher_exact_test(selected_pathways, gene_input,
  gene_pathway_matrix = "default")
\end{verbatim}
\end{Usage}
%
\begin{Arguments}
\begin{ldescription}
\item[\code{selected\_pathways}] A vecor of pathways to be used for enrichment analysis for genes in \emph{gene\_input}.It should have same ID types(E.g. pathway ID, pathway names) as the pathways in \emph{gene\_pathway\_matrix}.

\item[\code{gene\_input}] A vecor of genes to be annotated. It should have same ID types(E.g. Ensembl ID, HUGO gene symbol) as the genes in \emph{gene\_pathway\_matrix}.

\item[\code{gene\_pathway\_matrix}] A binary background matrix whose columns are the pathways/gene sets and
whose rows are all the genes from pathways/gene sets . It could be in sparse matrix format ((inherit from class "sparseMatrix" as in package Matrix) to save memory.
For gene i and pathway j, the value of matrix(i,j) is 1 is gene i belonging to pathway j otherwise 0.
Users could leave it as default value("default") so it will use pre-collected gene\_pathway\_matrix from GO Ontology and REACTOME databaase.
Otherwise, they could use their own customized gene\_pathway\_matrix
\end{ldescription}
\end{Arguments}
%
\begin{Value}
A list of two elements:
\begin{itemize}

\item selected\_pathways\_fisher\_pvalue - Fisher exact pvalue for selected pathways
\item selected\_pathways\_num\_genes - The number of genes for selected pathways in background

\end{itemize}

\end{Value}
%
\begin{Examples}
\begin{ExampleCode}
a=fisher_exact_test(selected_pathways=c("GO:0007250","GO:0008625"),gene_input=c("TRPC4AP","CDC37","TNIP1","IKBKB","NKIRAS2","NFKBIA","TIMM50","RELB","TNFAIP3","NFKBIB","HSPA1A","NFKBIE","SPAG9","NFKB2","ERLIN1","REL","TNIP2","TUBB6","MAP3K8"),gene_pathway_matrix="default")
\end{ExampleCode}
\end{Examples}
\inputencoding{utf8}
\HeaderA{from\_id2name}{from\_id2name}{from.Rul.id2name}
%
\begin{Description}\relax
If you use the default pathway databases(GO Ontologyand REACTOME),this function can help you to get pathways names from pathways IDs.
\end{Description}
%
\begin{Usage}
\begin{verbatim}
from_id2name(selected_pathways)
\end{verbatim}
\end{Usage}
%
\begin{Arguments}
\begin{ldescription}
\item[\code{selected\_pathways}] A list of GO and/or REACTOME pathways IDs. Each elmment is this list can be a single id or multi-ids seperated "\#"
\end{ldescription}
\end{Arguments}
%
\begin{Value}
A list of GO sub-root or REACTOME root names for provided pathways.
\end{Value}
%
\begin{Examples}
\begin{ExampleCode}
from_id2name((selected_pathways=list(c("GO:0032991#GO:0044425#GO:0044464"),"R-HSA-5357801")))
\end{ExampleCode}
\end{Examples}
\inputencoding{utf8}
\HeaderA{gene\_pathway\_matrix}{Homo sapiens GO ontology and REACTOME ontology gene-pathway realtionship}{gene.Rul.pathway.Rul.matrix}
\keyword{datasets}{gene\_pathway\_matrix}
%
\begin{Description}\relax
A rds R object contains GO ontology and REACTOME ontology gene-pathway realtionship
\end{Description}
%
\begin{Usage}
\begin{verbatim}
readRDS(system.file("extdata", "gene_pathway_matrix.rds", package = "GENEMABR"))
\end{verbatim}
\end{Usage}
%
\begin{Format}
Formal class 'dgCMatrix' [package "Matrix"]
\end{Format}
%
\begin{Source}\relax
http://geneontology.org/docs/download-ontology/, https://reactome.org/download-data
A binary matrix whose columns are the pathways/gene sets from GO ontology and REATOME database and 
whose rows are all the genes(represented by gene HUGO gene symbols) from  GO ontology and REATOME database.
For gene i and pathway j, the value of matrix(i,j) is 1 is gene i belonging to pathway j otherwise 0
\end{Source}
\inputencoding{utf8}
\HeaderA{get\_steps}{get\_steps}{get.Rul.steps}
%
\begin{Description}\relax
If you use the default pathway databases(GO Ontologyand REACTOME),this function allows you to extract  the distances from ceatain pathways to  GO roots or REACTOME roots nodes.
\end{Description}
%
\begin{Usage}
\begin{verbatim}
get_steps(selected_pathways)
\end{verbatim}
\end{Usage}
%
\begin{Arguments}
\begin{ldescription}
\item[\code{selected\_pathways}] A vecor of GO and/or REACTOME pathways IDs.
\end{ldescription}
\end{Arguments}
%
\begin{Value}
A list contains distances from pathways to GO root or REACTOME root nodes
\end{Value}
%
\begin{Examples}
\begin{ExampleCode}
get_steps(selected_pathways=c("GO:0005834","R-HSA-111469"))
\end{ExampleCode}
\end{Examples}
\inputencoding{utf8}
\HeaderA{human\_go\_ontology}{Homo sapiens GO ontology tree}{human.Rul.go.Rul.ontology}
\keyword{datasets}{human\_go\_ontology}
%
\begin{Description}\relax
A rds R object contains GO ontology relationships (tree structure)
\end{Description}
%
\begin{Usage}
\begin{verbatim}
readRDS(system.file("extdata", "human_go_ontology.rds", package = "GENEMABR"))
\end{verbatim}
\end{Usage}
%
\begin{Format}
Directed igraph format
\end{Format}
%
\begin{Details}\relax
The igraph format tree was constructed by using data from http://geneontology.org/docs/download-ontology/ (May 2108)
It has three root notes representing Molecular Function,Cellular Component and Biological Process (http://geneontology.org/docs/ontology-documentation/)
\end{Details}
%
\begin{Source}\relax
http://geneontology.org/docs/download-ontology/
\end{Source}
\inputencoding{utf8}
\HeaderA{human\_go\_roots}{human\_go\_roots}{human.Rul.go.Rul.roots}
\keyword{datasets}{human\_go\_roots}
%
\begin{Description}\relax
A rds R object contains GO ontology tree root nodes
\end{Description}
%
\begin{Usage}
\begin{verbatim}
readRDS(system.file("extdata", "human_go_roots.rds", package = "GENEMABR"))
\end{verbatim}
\end{Usage}
%
\begin{Format}
A vector of GO ontology root notes (ID)
\end{Format}
%
\begin{Details}\relax
You can view tree stuctor of GO ontology at  https://www.ebi.ac.uk/QuickGO/
Thre are three roots notes in GO ontology tree: GO:0008150 (biological\_process), GO:0003674(molecular\_function), GO:0005575(cellular\_component)
\end{Details}
%
\begin{Source}\relax
http://geneontology.org/docs/download-ontology/
\end{Source}
\inputencoding{utf8}
\HeaderA{human\_go\_sub\_roots}{human\_go\_sub\_roots}{human.Rul.go.Rul.sub.Rul.roots}
\keyword{datasets}{human\_go\_sub\_roots}
%
\begin{Description}\relax
A rds R object contains GO ontology tree sub-root nodes (The children of root nodes).
\end{Description}
%
\begin{Usage}
\begin{verbatim}
readRDS(system.file("extdata", "human_go_sub_roots.rds", package = "GENEMABR"))
\end{verbatim}
\end{Usage}
%
\begin{Format}
A list of three elements contains GO ontology sub-root notes (ID)/the children of three root notes
\end{Format}
%
\begin{Details}\relax
You can view tree stuctor of GO ontology at  https://www.ebi.ac.uk/QuickGO/
\end{Details}
%
\begin{Source}\relax
http://geneontology.org/docs/download-ontology/
\end{Source}
\inputencoding{utf8}
\HeaderA{human\_reactome\_ontology}{Homo sapiens REACTOME ontology tree}{human.Rul.reactome.Rul.ontology}
\keyword{datasets}{human\_reactome\_ontology}
%
\begin{Description}\relax
A rds R object contains Reactome ontology relationships (tree structure)
\end{Description}
%
\begin{Usage}
\begin{verbatim}
readRDS(system.file("extdata", "human_reactome_ontology.rds", package = "GENEMABR"))
\end{verbatim}
\end{Usage}
%
\begin{Format}
Directed igraph format
\end{Format}
%
\begin{Details}\relax
The igraph format tree was constructed by using  data from https://reactome.org/download-data (May 2108)
It has several root nodes representing REACTOME pathway categories (https://reactome.org/PathwayBrowser/)
\end{Details}
%
\begin{Source}\relax
https://reactome.org/download-data
\end{Source}
\inputencoding{utf8}
\HeaderA{human\_reactome\_roots}{human\_reactome\_roots}{human.Rul.reactome.Rul.roots}
\keyword{datasets}{human\_reactome\_roots}
%
\begin{Description}\relax
A rds R object contains REACTOME tree root nodes
\end{Description}
%
\begin{Usage}
\begin{verbatim}
readRDS(system.file("extdata", "human_reactome_roots.rds", package = "GENEMABR"))
\end{verbatim}
\end{Usage}
%
\begin{Format}
A vector of REACTOME root notes (ID)
\end{Format}
%
\begin{Details}\relax
You can view tree stuctor of REACTOME at https://reactome.org/PathwayBrowser/
\end{Details}
%
\begin{Source}\relax
https://reactome.org/download-data
\end{Source}
\inputencoding{utf8}
\HeaderA{regression\_selected\_pathways}{regression\_selected\_pathways}{regression.Rul.selected.Rul.pathways}
%
\begin{Description}\relax
This function allows you to extracte enriched pathways for gene module/list via regressioin (elastic net) based method
\end{Description}
%
\begin{Usage}
\begin{verbatim}
regression_selected_pathways(gene_input, gene_pathway_matrix = "default",
  lambda = 0.007956622, alpha = 0.5, ...)
\end{verbatim}
\end{Usage}
%
\begin{Arguments}
\begin{ldescription}
\item[\code{gene\_input}] A vecor of genes to be annotated. It should have same ID types(Ensembl ID, HUGO gene symbol) as the genes in \emph{gene\_pathway\_matrix}.

\item[\code{gene\_pathway\_matrix}] A binary background matrix whose columns are the pathways/gene sets and 
whose rows are all the genes from pathways/gene sets . It could be in sparse matrix format ((inherit from class "sparseMatrix" as in package Matrix) to save memory.
For gene i and pathway j, the value of matrix(i,j) is 1 is gene i belonging to pathway j otherwise 0.
Users could leave it as default value("default") so it will use pre-collected gene\_pathway\_matrix from GO Ontology and REACTOME databaase.
Otherwise, they could use their own customized gene\_pathway\_matrix

\item[\code{lambda}] We use glmnet function to do regression. \emph{lambda} is an argument in \strong{glmnet}. See \strong{glmnet} function for more details
Here we use default value 0.007956622 after preliminary study. It can be overridden by giving \emph{nlambda} and \emph{lambda.min.ratio arguments}.

\item[\code{alpha}] The elasticnet mixing parameter, with 0≤α≤ 1. The penalty is defined as
(1-α)/2||β||\_2\textasciicircum{}2+α||β||\_1.
alpha=1 is the lasso penalty, and alpha=0 the ridge penalty. Default value: 0.5.

\item[\code{...}] Other paramaters for glmnet function.
\end{ldescription}
\end{Arguments}
%
\begin{Value}
A list of four elements: 
\begin{itemize}

\item selected\_pathways\_names - Pathways names for selected pathways
\item selected\_pathways\_coef - Regression coefficients value for selected pathways
\item selected\_pathways\_fisher\_pvalue - Fisher exact pvalue for selected pathways
\item selected\_pathways\_num\_genes - The number of genes for selected pathways in background

\end{itemize}

\end{Value}
%
\begin{Examples}
\begin{ExampleCode}
a=regression_selected_pathways(gene_input =c("TRPC4AP","CDC37","TNIP1","IKBKB","NKIRAS2","NFKBIA","TIMM50","RELB","TNFAIP3","NFKBIB","HSPA1A","NFKBIE","SPAG9","NFKB2","ERLIN1","REL","TNIP2","TUBB6","MAP3K8"),gene_pathway_matrix="default",lambda=0.007956622,alpha=0.5)
\end{ExampleCode}
\end{Examples}
\printindex{}
\end{document}
